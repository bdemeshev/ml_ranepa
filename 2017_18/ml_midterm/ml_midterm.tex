\documentclass[11pt]{article}

\usepackage{tikz} % картинки в tikz
\usepackage{microtype} % свешивание пунктуации

\usepackage{array} % для столбцов фиксированной ширины

\usepackage{indentfirst} % отступ в первом параграфе

\usepackage{sectsty} % для центрирования названий частей
\allsectionsfont{\centering}

\usepackage{verbatim}
\usepackage{amsmath} % куча стандартных математических плюшек

\usepackage[top=1.4cm, left=1cm, right=1cm, bottom=1.5cm]{geometry} % размер текста на странице

\usepackage{lastpage} % чтобы узнать номер последней страницы

\usepackage{enumitem} % дополнительные плюшки для списков
%  например \begin{enumerate}[resume] позволяет продолжить нумерацию в новом списке
\usepackage{caption}


\usepackage{fancyhdr} % весёлые колонтитулы
\pagestyle{fancy}
\lhead{Машинное обучение}
\chead{}
\rhead{2018-01-24, зачёт}
\lfoot{}
\cfoot{}
\rfoot{\thepage/\pageref{LastPage}}
\renewcommand{\headrulewidth}{0.4pt}
\renewcommand{\footrulewidth}{0.4pt}



\usepackage{todonotes} % для вставки в документ заметок о том, что осталось сделать
% \todo{Здесь надо коэффициенты исправить}
% \missingfigure{Здесь будет Последний день Помпеи}
% \listoftodos --- печатает все поставленные \todo'шки


% более красивые таблицы
\usepackage{booktabs}
% заповеди из докупентации:
% 1. Не используйте вертикальные линни
% 2. Не используйте двойные линии
% 3. Единицы измерения - в шапку таблицы
% 4. Не сокращайте .1 вместо 0.1
% 5. Повторяющееся значение повторяйте, а не говорите "то же"



\usepackage{fontspec}
\usepackage{polyglossia}

\setmainlanguage{russian}
\setotherlanguages{english}

% download "Linux Libertine" fonts:
% http://www.linuxlibertine.org/index.php?id=91&L=1
\setmainfont{Linux Libertine O} % or Helvetica, Arial, Cambria
% why do we need \newfontfamily:
% http://tex.stackexchange.com/questions/91507/
\newfontfamily{\cyrillicfonttt}{Linux Libertine O}

\AddEnumerateCounter{\asbuk}{\russian@alph}{щ} % для списков с русскими буквами
\setlist[enumerate, 2]{label=\asbuk*),ref=\asbuk*}

%% эконометрические сокращения
\DeclareMathOperator{\Cov}{Cov}
\DeclareMathOperator{\Corr}{Corr}
\DeclareMathOperator{\Var}{Var}
\DeclareMathOperator{\E}{E}
\def \hb{\hat{\beta}}
\def \hs{\hat{\sigma}}
\def \htheta{\hat{\theta}}
\def \s{\sigma}
\def \hy{\hat{y}}
\def \hY{\hat{Y}}
\def \v1{\vec{1}}
\def \e{\varepsilon}
\def \he{\hat{\e}}
\def \z{z}
\def \hVar{\widehat{\Var}}
\def \hCorr{\widehat{\Corr}}
\def \hCov{\widehat{\Cov}}
\def \cN{\mathcal{N}}


\begin{document}




\begin{enumerate}

  \item Предположим, что $A$ — матрица констант, а $X$ — матрица переменных, найдите $d(XAX)$. Как упростится ответ, если $AX=XA$?

  \item Задана зависимость $y_i = \beta x_i + u_i$, ошибки $u_i$ нормальны $\cN(0;1)$. Исследователь Василий использует следующий способ построения прогнозов: $\hat y_f = \gamma \cdot \hb x_f$, где $\hb$ — это оценка МНК, а $\gamma$ — некоторая константа. При каком $\gamma$ ожидаемый квадрат ошибки прогноза будет минимальным? Как на практике подобрать такое $\gamma$?

  \item Как изменится энтропия дискретной величины $X$, если величину домножить на 20? А если у величины $X$ есть функция плотности?

  \item 
Постройте классификационное дерево для прогнозирования $y$ с помощью $x$ и $z$ на обучающей выборке:

\begin{tabular}{lrrrrr}
\toprule
$x_i$ & $0$ & $0$ & $0$ & $1$ & $1$ \\
$z_i$ & $1$ & $2$ & $3$ & $4$ & $5$ \\
$y_i$ & $0$ & $1$ & $1$ & $0$ & $0$ \\
\bottomrule
\end{tabular}

Критерий деления узла на два — минимизация индекса Джини. Дерево строится до идеальной классификации.

\item 
На плоскости расположены колонии рыжих и чёрных муравьёв. Рыжих колоний три и они имеют координаты $(-1, 1)$, $(1, -1)$ и $(1, 1)$. Чёрных колоний одна и она имеет координаты $(0, 0)$.

\begin{enumerate}
  \item Поделите плоскость на «зоны влияния» рыжих и чёрных используя метод одного и трёх ближайших соседей.
  
\item С помощью кросс-валидации с выкидыванием отдельных наблюдений выберите оптимальное число соседей $k$ перебрав $k \in \{1, 3 \}$. Целевой функцией является количество несовпадающих прогнозов.
\end{enumerate}

\item 

Машин-лёрнер Василий лично раздобыл выборку из четырёх наблюдений.


\begin{tabular}{rrrrr}
\toprule
$x_i$ & 1 & 2 & 3 & 4 \\
$y_i$ & 6 & 6 & 12 & 18 \\
\bottomrule
\end{tabular}

Два готовых дерева для бустинга Василий подглядел у соседа:

\begin{minipage}{0.4\textwidth}
\begin{center}
%\begin{tikzpicture}[scale = 0.025]
\includegraphics[scale=0.1]{images/two_trees.png}
%\input{armada.tikz}
%\end{tikzpicture}
\end{center}
\end{minipage}

Василий решил использовать бустинг c темпом обучение $\eta$. Прогнозы в каждом листе конкретного дерева Василий строит минимизируя функцию:
\[
	Q = \sum_{i=1}^n (y_i - \hat y_i)^2 + \lambda \sum_{j=1}^T w_j^2,
\]
где $y_i$ — прогнозируемое значение для $i$-го наблюдения, $n$ — количество наблюдений, $w_j$ — прогноз в $j$-ом листе, $T$ — количество листов на дереве.


Какие прогнозы внутри обучающей выборки получит Василий при $\eta=1$ и $\lambda=0.5$?


\end{enumerate}

\end{document}
